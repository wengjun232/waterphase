% !TeX root = ../thuthesis-example.tex

\chapter{The JUNO detector}
The Jiangmen Underground Neutrino Observatory~(JUNO) is a multipurpose neutrino experiment under construction in southern China. The primary scientific objective of JUNO is to determine the neutrino mass ordering and precisely measure neutrino oscillation parameters by detecting reactor antineutrinos from the Yangjiang and Taishan nuclear power plants, located approximately \SI{53}{km} from the JUNO site. The JUNO detector consists of a central detector~(CD) containing \SI{20}{kilotons} of liquid scintillator~(LS), a water Cherenkov detector~(WCD) serving as a cosmic muon veto, and a top tracker~(TT) for additional muon tracking. The CD is housed within a large acrylic sphere with a diameter of \SI{35.4}{m}, which is further enclosed by a stainless steel lattice structure. The WCD surrounds the CD and is filled with \SI{35}{kilotons} of ultrapure water, while the TT is positioned above the WCD. The entire detector assembly is situated approximately \SI{700}{m} underground, equivalent to \SI{1800}{m} of water overburden, to mitigate cosmic ray interference.

