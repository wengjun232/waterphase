
\section{The likelihood-based reconstruction of the JUNO water-phase}
Super-Kamiokande~(SK) is the largest water Cherenkov detector in the world, with a water mass of 50 kilotons~\cite{SK}.
Building upon the liquid scintillator-Cherenkov combined track reconstruction technique developed for the MiniBooNE experiment~\cite{minibone}, SK collaboration has advanced a likelihood-based reconstruction method, utilizing PMT charge and time information~\cite{SKfiTQun}, named as fiTQun.  For JUNO water-phase, we have implemented targeted improvements to the fiTQun and extended its application to low-energy event reconstruction at the MeV scale.
\subsection{The Likelihood function}
FiTQun
\subsection{The Fast reconstruction of the JUNO water-phase}
\section{The evaluation of reconstructions}
\subsection{In the dector simulation}
\subsection{In the electronic simulation}
\subsection{The \ce{AmC} and \ce{AmBe} calibrations in water-phase}
In Run~3671, the \ce{AmBe} source is placed at $(0,0,-10)$\si{m}.

\subsection{The combination of the two reconstruction algorithms}
\subsubsection{The events selected from the fast reconstruction}
